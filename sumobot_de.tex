\documentclass[a4paper,12pt]{article}
\usepackage[ngerman]{babel}
\usepackage{ucs}
\usepackage{multirow}
\usepackage{xltxtra}
\usepackage[utf8x]{inputenc}
\usepackage{fontspec}
\usepackage[automark]{scrpage2}
\usepackage{eurosym}
\usepackage{graphicx}
\usepackage[paper=a4paper,left=25mm,right=25mm,top=25mm,bottom=25mm]{geometry}
\pagestyle{scrheadings}
\setmainfont[Mapping=tex-text]{Liberation Serif}
\clearscrheadfoot
\ohead{Regelstand: \today}
\title{2020 SUMO Challenge Regeln}
\makeatletter
\let\inserttitle\@title
\makeatother

\begin{document}

 \begin{center}
\includegraphics[width=0.5\textwidth]{logo.png}

\huge                      % Schriftgröße einstellen
\bfseries                   % Fettdruck einschalten
\inserttitle
  \end{center}
  %Inhaltliche Änderungen im Vergleich zu den Regeln von 2017 sind \textbf{fett} markiert. Im Zweifel ist die Interpretation der Regeln durch die Schiedsrichter bindend.
\section{Ziel}
Entwurf, Bau und Programmierung eines autonomen Roboters, der einen gegnerischen Sumoroboter suchen und aus einem erhöhten Ring schieben kann.
\section{Divisionen/Massenklassen}
Bitte entnehmen Sie der untenstehenden Tabelle, in welcher Division/Massenklasse Sie teilnehmen möchten.
\begin{center}
\begin{tabular}{|c|c|c|} \hline
	\multirow{2}*{1kg} & \multirow{2}*{2kg} & Lego Only 1Kg \\
	& & (Optional) \\ \hline
	%ES &  & ES \\ \hline
	%MS & ** & MS \\ \hline
	MS & MS & MS \\ \hline
	HS & HS & HS \\ \hline
	%** & UP & ** \\ \hline
\end{tabular} \\ \vspace{\baselineskip}
\end{center}

\section{Roboter}
Autonomer Roboter, jede beliebige Plattform, die 1.500 USD oder weniger kostet und die folgenden Designbeschränkungen erfüllt, die beim Robot Check-In überprüft werden.
\begin{itemize}
	\item Die maximale Größe des Roboters beträgt 25 cm x 18 cm ohne Höhenbeschränkung, gemessen mit allen beweglichen Komponenten in ihrer Ausgangsposition.
	\item TODO \ldots
\end{itemize}

\end{document}
