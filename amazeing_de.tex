\documentclass[a4paper,12pt]{article}
\usepackage[ngerman]{babel}
\usepackage{ucs}
\usepackage{multirow}
\usepackage{xltxtra}
\usepackage[utf8x]{inputenc}
\usepackage{fontspec}
\usepackage[automark]{scrpage2}
\usepackage{eurosym}
\usepackage{graphicx}
\usepackage[paper=a4paper,left=25mm,right=25mm,top=25mm,bottom=25mm]{geometry}
\usepackage[normalem]{ulem}
\pagestyle{scrheadings}
\setmainfont[Mapping=tex-text]{Liberation Serif}
\clearscrheadfoot
\begin{document}
\ohead{Regelstand: \today}
\title{Regeln a-MAZE-ing Challenge 2020}
\makeatletter
\let\inserttitle\@title
\makeatother

 \begin{center}
\includegraphics[width=0.5\textwidth]{logo.png}

\huge                      % Schriftgröße einstellen
\bfseries                   % Fettdruck einschalten
\inserttitle
  \end{center}
  Inhaltliche Änderungen im Vergleich zu den Regeln von 2018 sind \textbf{fett} markiert. Im Zweifel ist die Interpretation der Regeln durch die Schiedsrichter bindend.
\section{Aufgabe}
Baue und programmiere einen Roboter, der ohne herunterzufallen einer Strecke aus Holzstücken folgen
kann. Je schneller die Strecke aus Holzstücken abgefahren werden kann, desto mehr Punkte können erreicht
werden.
\section{Wer kann teilnehmen?}
Teams von \emph{2 bis 4 Spielern} in \emph{folgenden Altersgruppen}:
\begin{itemize}
	\item Altersgruppe 1 (Middle School): 10-13 Jahre
	\item Altersgruppe 2 (High School): 14-17 Jahre
\end{itemize}
\section{Materialanforderungen}
Autonomer Roboter basierend auf jeglicher Plattform, der maximal  \euro{ 1500} kostet und den folgenden
Designanforderungen, die beim \emph{Check-In überprüft werden}, entspricht:
\begin{itemize}
\item Es dürfen \emph{keinerlei Sensoren} verwendet werden. Hierbei zählen auch Gyroskope als nicht erlaubte Sensoren. Die Benutzung von Tachowerten ist erlaubt.
\item \sout{Der Roboter kann einem 46 cm langen geraden Holzstück folgen, eine 90°-Rechtskurve fahren und dann wieder
einem 46 cm langem geraden Holzstück folgen.}
\item Das Volumen des Roboter darf 65030 cm$^{3}$ \emph{nicht} überschreiten.
\end{itemize}
\section{Spielregeln}
\begin{itemize}
\item Der Roboter hat \emph{2 Minuten} um die Strecke abzufahren.
\item Innerhalb dieser Zeit können die Teams beliebig oft versuchen die Strecke abzufahren.
\textbf{Ein neuer Versuch kann jedoch erst gestartet werden wenn der Roboter von der Strecke
heruntergefallen ist.}
\item \textbf{Ein Roboter zählt als heruntergefallen wenn irgendeines der Räder die Oberfläche der Strecke
vollständig nicht mehr berührt.}
\item Gewertet wird der Versuch mit der höchsten Punktzahl.
\end{itemize}
\section{Spielfeld}
Die a-MAZE-ing-Tracks einer Altersgruppe sind alle identisch und aus 24 cm breiten und 2 cm starken
Holzstücken aufgebaut. Es gibt zahlreiche verschieden lange Holzbretter mit Winkeln von 45, 90 und 135
Grad. \textbf{Über die erlaubten Kombinationen gibt unteres Bild Auskunft. Insbesondere darf ein
90-Grad-Winkel nicht unter Zuhilfenahme eines dreieckigen Verbinders gebildet werden, dreieckige 
Verbinder dürfen nicht aneinander angrenzen und Verbinder müssen stets so ausgerichtet sein, dass
jede Kante des Dreiecks entweder vollständig oder gar nicht an ein Holzstück angrenzt.}
\textbf{Die Holzstücke werden im Normalfall mit Panzerklebeband (duct tape) aneinander befestigt.
Es können jedoch auch andere Befestigungsarten wie z.B. Kleber oder Schrauben verwendet werden.
Im Allgemeinen wird versucht die Strecke soweit möglich frei von Unregelmäßigkeiten zu halten.}
\textbf{Die Schiedsrichter dürfen entscheiden ob die Strecken für die einzelnen Altersgruppen
getrennt werden oder ob die Ziellinien der verschiedenen Altersgruppen lediglich durch Markierungen
kenntlich gemacht werden.}
\begin{itemize}
\item Altersgruppe 1 (Middle School) – Ziellinie ist in der Mitte zwischen der fünften und der sechsten Abzweigung
\item Altersgruppe 2 (High School) – Ziellinie ist am Ende der Strecke
\end{itemize}
\section{Wertungszeitraum}
\par Die Art der Wertung der einzelnen Ergebnisse im Bezug auf den gesamten Wettbewerb wird am ersten Wettbewerbstag von den Schiedsrichtern bekanntgegeben. Änderungen der Wertungsmodalitäten bleiben den Schiedsrichtern vorbehalten.
\section{Punktevergabe}
\begin{itemize}
\item Jede gefahrene gerade Teilstrecke wird mit 50 Punkten bewertet. Jede gefahrene Kurve 
wird mit 100 Punkten gewertet.
\item \textbf{Eine Strecke gilt als gefahren wenn irgendein Teil des Roboters die Linie am Ende
dieser Strecke überquert hat.}
\item Wenn der Roboter vor der Ziellinie von den Holzstücken abkommt, zählt dies als
abgeschlossener Versuch mit den Teilpunkten nach untenstehender Tabelle. Wenn noch Zeit übrig ist
können die Spieler einen erneuten Durchlauf versuchen um ein besseres Ergebnis zu erzielen.
Der Zeit-Bonus für den Rest der Zeit  wird nur vergeben wenn der Roboter die Ziellinie überquert hat.
\item \emph{Zeit-Bonus-Punkte} werden nur erzielt, wenn der Roboter die Ziellinie in unter 120
Sekunden erreicht. Jede verbleibende Sekunde zählt als 1 Punkt. Der Zeit-Bonus wird zu den
erzielten Punkten addiert.
\end{itemize}
\section{Punktetabelle}
\begin{center}
\begin{tabular}{|c|c|c|c|c|c|c|c|c|} \hline
	\multirow{2}*{Altersgruppe} & 1. & 1. & 2. & 2. & 3. & 3. & 4. \\
	& Strecke & Kurve & Strecke & Kurve & Strecke & Kurve & Strecke   \\ \hline
	MS (10-13) & 50 & 100 & 50 & 100 & 50 & 100 & 50   \\ \hline
	HS (14-17) & 50 & 100 & 50 & 100 & 50 & 100 & 50  \\ \hline
\end{tabular} \\ \vspace{\baselineskip}
\begin{tabular}{|c|c|c|c|c|c|c|c|c|} \hline
	\multirow{2}*{Altersgruppe} & 4. & 5. & 5. & 6. & 6. & 7. & \multirow{2}*{Gesamt} \\
	& Kurve & Strecke & Kurve & Strecke & Kurve & Strecke  &   \\ \hline
	MS (10-13) & 100 & 50 & 100 & 50 & & & 800 \\ \hline
	HS (14-17) & 100 & 50 & 100 & 50 & 100 & 50  & 950 \\ \hline
\end{tabular}
\end{center}
\end{document}
