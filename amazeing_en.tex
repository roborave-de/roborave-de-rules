\documentclass[a4paper,12pt]{article}
%\usepackage[ngerman]{babel}
\usepackage{ucs}
\usepackage{multirow}
\usepackage{xltxtra}
\usepackage[utf8x]{inputenc}
\usepackage{fontspec}
\usepackage[automark]{scrpage2}
\usepackage{eurosym}
\usepackage{graphicx}
\usepackage[paper=a4paper,left=25mm,right=25mm,top=25mm,bottom=25mm]{geometry}
\pagestyle{scrheadings}
\setmainfont[Mapping=tex-text]{Liberation Serif}
\clearscrheadfoot
\begin{document}
\ohead{Last edit: \today}
\title{Rules a-MAZE-ing Challenge 2018}

 \begin{center}
\includegraphics[width=0.5\textwidth]{logo.png}

\huge \bfseries Rules a-MAZE-ing Challenge 2018
  \end{center}
  This is only an unofficial translation. In case of doubt, only the newest official version of the German rules will
  count.
  %Inhaltliche Änderungen im Vergleich zu den Regeln von 2017 sind \textbf{fett} markiert. Im Zweifel ist die Interpretation der Regeln durch die Schiedsrichter bindend.
\section{Goal}
To design, build, and program a robot that can follow a raised wooden maze without falling off. The faster you
can complete the maze increases your overall score.
\section{Who Can Play}
Teams of \emph{2 to 4 players} in \emph{separate divisions} for:
\begin{itemize}
	\item Middle School (Age 10-13)
	\item High School (Age 14-17)
\end{itemize}
\section{Required Materials}
Autonomous robot, any platform, costing \euro{ 1500} or less, and meets the following design constraints,
which will be verified during
 \emph{Check-In}:
\begin{itemize}
\item Robot is not allowed to use any sensors to assist it in following the maze, wheel encoders are allowed.
\item Robot can navigate a simple 46 cm straight, 90 degree right, 46 cm straight maze during check in.
\item Volume of the robot must \emph{not} exceed 65030 cubic cm.
\end{itemize}
\section{General Rules of Play}
\begin{itemize}
\item The robot has 2 minutes to complete the maze, the clock runs backwards from 120 seconds.
\item Within these 2 minutes you can do as many trials as you want. Only the trial with the highest
number of points will be counted
\end{itemize}
\section{Challenge Specifications}
The a-MAZE-ing tracks will all be identical and are constructed of wood that is approximately 24 cm wide and 2
cm tall. There are various lengths, as determined by the division, with angles of any combination of 45, 90, and
135 degrees that can turn in either direction.
While all two divisions will utilize the same track, each division has a different finish line:
.
\begin{itemize}
\item Middle School Division – Finish line will be halfway between the 5th and 6th angled turn
\item High School Division – Finish line is at the end of the track
\end{itemize}
\section{Scoring Period}
\par The way the earned points are counted for the challenge will be announced on the first day of the event.
\section{Scoring}
\begin{itemize}
\item Each completed straight-away is worth 50 points and each completed angle is worth 100 points.
\item If the robot falls off the maze before reaching the finish line, then the run is concluded, and the score received
includes any portion of the maze that is completed in it’s entirety, but no time bonus points are awarded.
\item emph{Time bonus} points are only awarded if the robot reaches the finish line before the 120 seconds ends. Any
remaining time is then added to the maze score as a time bonus point value (the whole number part of the
time).
\end{itemize}
\section{Scoring Matrix}
\begin{center}
\begin{tabular}{|c|c|c|c|c|c|c|c|c|} \hline
	\multirow{2}*{Division} & 1. & 1. & 2. & 2. & 3. & 3. & 4. \\
	& Straight & Turn & Straight &Turn & Straight &Turn & Straight  \\ \hline
	MS (10-13) & 50 & 100 & 50 & 100 & 50 & 100 & 50   \\ \hline
	HS (14-17) & 50 & 100 & 50 & 100 & 50 & 100 & 50  \\ \hline
\end{tabular} \\ \vspace{\baselineskip}
\begin{tabular}{|c|c|c|c|c|c|c|c|c|} \hline
	\multirow{2}*{Division} & 4. & 5. & 5. & 6. & 6. & 7. & Total \\
	& Turn & Straight & Kurve & Straight & Turn & Straight  & Score  \\ \hline
	MS (10-13) & 100 & 50 & 100 & 50 & & & 800 \\ \hline
	HS (14-17) & 100 & 50 & 100 & 50 & 100 & 50  & 950 \\ \hline
\end{tabular}
\end{center}
\end{document}